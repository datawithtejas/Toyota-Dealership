\documentclass{article}
\usepackage{graphicx}
\usepackage{float}
\usepackage{geometry}
\geometry{a4paper, margin=1in}

\usepackage{Sweave}
\begin{document}
\input{toyota_research-concordance}

\title{Toyota Dealership: Operational Efficiency \& Profit Strategy}
\author{Analyst Report}
\maketitle


\section{Report Objective}

This report serves as a strategic diagnostic tool designed to transform raw sales data into actionable business intelligence. By analyzing \textbf{40,000 historical transactions}, this document aims to move beyond simple volume metrics and evaluate the true operational health of the dealership network.

Upon review of this report, stakeholders will be equipped with the following insights:

\begin{itemize}
    \item \textbf{Profitability Architecture:} A clear breakdown of where profit is actually generated—distinguishing between "Volume Leaders" (who win on quantity) and "Efficiency Leaders" (who win on margin).
    
    \item \textbf{Inventory Velocity Analysis:} A statistical evaluation of the "Days on Lot" metric to determine if current holding costs are eroding margins and to validate if an "Age-Based" pricing strategy is required.
    
    \item \textbf{Consumer Behavior Trends:} A definitive answer to the "Cash vs. Finance" debate, identifying if payment methods or vehicle segments (e.g., SUVs vs. Sedans) dictate profitability.
    
    \item \textbf{Valuation Precision:} A data-driven Pricing Model (Regression Analysis) that isolates the exact dollar value of vehicle age and mileage, allowing for automated, risk-free inventory appraisal.
\end{itemize}

Ultimately, this report provides the statistical foundation necessary to standardize operations across Texas, New York, and California, ensuring a consistent and scalable business model.

\section{Data}



The Dataset used was generated synthetically with the help of chatgpt.This dataset serves as a comprehensive operational record for Toyota retail performance, capturing \textbf{40,000 individual sales transactions} across major dealerships in Texas, New York, and California. It provides a 360-degree view of the sales cycle by combining vehicle specifics such as \textbf{Make, Model, Year, Mileage, and Car Type} with critical financial metrics like \textbf{Sale Price, Cost Price, and Net Profit}. Beyond basic accounting, the data enables deep strategic analysis through operational variables like \textbf{Days on Lot}, \textbf{Payment Type} (Cash, Loan, Lease), and \textbf{Customer Type} (New vs. Returning), making it an ideal resource for auditing inventory efficiency and regional profitability trends.

\section{Data Overview}

\begin{center}
    \fbox{
      \begin{minipage}{0.95\textwidth}
        \begin{center}
           
            \includegraphics[width=\textwidth]{dataset_infographic_summary.png}
        \end{center}
      \end{minipage}
    }
\end{center}
\section{Profitability Analysis}

\subsection{Total Profit by Dealer Location}
\begin{Schunk}
\begin{Soutput}
# A tibble: 6 × 5
  dealer              Total_Profit Avg_Profit_Per_Car Avg_Days_On_Lot Units_Sold
  <fct>                      <int>              <dbl>           <dbl>      <int>
1 Toyota of Houston       13691712              3014.            39.7       4543
2 Brooklyn Toyota         13575742              3013.            39.5       4506
3 Toyota of Los Ange…     13525227              3039.            39.5       4451
4 Queens Toyota           13502928              2999.            39.8       4502
5 Dallas Toyota           13463151              3045.            39.6       4422
6 San Diego Toyota        13249424              3015.            39.5       4394
\end{Soutput}
\end{Schunk}


This data shows a highly competitive landscape where \textbf{volume is currently beating margin.} While Houston is the top earner due to high sales volume, Dallas is actually the most efficient at squeezing profit out of every single car. There are 4 major takeaways:

\begin{enumerate}
    \item \textbf{Toyota of Houston - "Volume King"}
       
    The Toyota of Houston ranked \#1 with \$13.69 Million in Total Profit. They achieved this by moving the most metal (4,543 units). Houston's dominance is purely a volume play. Their marketing or market size is generating the most foot traffic, allowing them to overpower other dealers despite having "average" margins.
    
    \item \textbf{Dallas Toyota - Efficiency Leader}
    
    Ranked \#5 in Total Profit, but \#1 in Profit Per Car (\$3,044). They make roughly \$30--\$70 more profit on every single car than their competitors. Dallas likely has a more profitable inventory mix (e.g., selling more Tundras/Tacomas than Corollas) or a superior sales team that holds gross profit better during negotiations.
    
    \item \textbf{Toyota of Manhattan}
    
    Manhattan Toyota has given us the lowest profit per car (\$2,973), so it is important to understand what is driving this anomaly. This is likely geographic. Manhattan buyers likely purchase smaller, lower-margin sedans or hybrids rather than high-margin trucks, naturally capping their efficiency.
    
    \item \textbf{Operational Consistency (The "39-Day Rule")}
    
    Every single dealership has an \texttt{Avg\_Days\_On\_Lot} between 39.5 and 39.8 days.
    
    \textbf{WHY THIS MATTERS}: This extreme consistency suggests corporate policy is dictating pricing adjustments. It implies we are systematically discounting cars at the 40-day mark to move them.
  
    \textbf{Opportunity}: If we can improve marketing to lower this to 35 days, we could turn inventory 10\% faster, potentially adding \$1M+ to the group's bottom line without buying more stock.
  
\end{enumerate}


\subsection{Profit Margins by Car Type}
\begin{center}
\includegraphics{toyota_research-004}
\end{center}

There is remarkable consistency across the board. The chart reveals a critical and somewhat counter-intuitive insight: \textbf{Vehicle "Size" is not correlating with "Margin."} Contrary to standard industry assumptions, your data proves that selling larger Trucks and SUVs does not generate significantly higher returns than selling Sedans or Hatchbacks.

\textbf{We should focus on three major findings:}
\begin{enumerate}
   \item \textbf{Margin Parity:} Median profit is consistent (\textasciitilde\$2,700--\$2,800) across all four categories, meaning a heavy-duty Tundra generates the same net return as a compact Yaris.
   \item \textbf{Uniform Risk Profile:} The volatility of profit is identical across segments; the risk of low-margin deals is equal whether stocking small cars or large trucks.
   \item \textbf{Universal Upside:} High-profit outliers ($>\$6,000$) appear in every category, proving that "home run" deals are segment-agnostic and just as achievable with Hatchbacks as with Trucks.
\end{enumerate}

Therefore we should \textbf{Prioritize Volume over Size}. Since shifting inventory toward larger vehicles yields no per-unit margin benefit, the dealership should abandon segment exclusivity. Strategy should pivot to turnover speed and maintaining a diverse inventory (including Hatchbacks and Sedans) to capture the widest possible customer base without sacrificing margin.
   
\subsection{Statistical Verification: ANOVA Test}

To statistically confirm the visual evidence from the boxplots, we performed a one-way ANOVA test to determine if the mean profit differs significantly across vehicle segments.

\begin{Schunk}
\begin{Soutput}
               Df    Sum Sq Mean Sq F value Pr(>F)
CarType         3 1.070e+07 3568296   1.953  0.119
Residuals   39996 7.308e+10 1827062               
\end{Soutput}
\end{Schunk}

The P-value represents the probability that the observed differences in profit are due to random chance. 
\begin{itemize}
    \item A \textbf{high P-value ($> 0.05$)} confirms that there is \textbf{no statistically significant difference} in profitability between Hatchbacks, Sedans, SUVs, and Trucks.
    \item This mathematically validates our strategy: \textbf{The dealership does not gain margin by prioritizing larger vehicles.}
\end{itemize}

\begin{Schunk}
\begin{Sinput}
> # 2. Levene's Test (Difference in Volatility/Risk)
> # This tests if one car type is "riskier" (more variable) than others
> leveneTest(Profit ~ CarType, data = df)
\end{Sinput}
\begin{Soutput}
Levene's Test for Homogeneity of Variance (center = median)
         Df F value Pr(>F)
group     3  0.6128 0.6066
      39996               
\end{Soutput}
\begin{Sinput}
> 
> 
\end{Sinput}
\end{Schunk}

\textbf{The "Uniform Risk" Validation}

Our statistical analysis returned a P-value of \textbf{0.61}, which is statistically significant in proving consistency. This result validates a critical business reality: \textbf{we have a Uniform Risk Profile.}

Executives often fear that shifting inventory toward different vehicle segments creates financial instability. However, this test proves that the volatility (risk) of profit is identical across all categories. 

\begin{itemize}
    \item \textbf{Implication for Inventory Strategy:} We do not need to "hedge" our inventory by stocking "safe" Sedans. A heavy-duty Truck carries no more margin risk than a compact Hatchback.
    \item \textbf{Forecasting Stability:} Because the profit variance is constant, our financial forecasting models can remain simple. We do not need complex risk adjustments based on the mix of cars on the lot; cash flow predictability remains stable regardless of whether we sell 100 Trucks or 100 Sedans.
\end{itemize}

\subsection{Avg profit by State}

\begin{center}
\includegraphics{toyota_research-007}

\end{center}

\subsection{ The "Border-Agnostic" Model}

Our regional profitability analysis returned a result of high strategic importance: \textbf{Geographic Irrelevance.}

As displayed in the chart, the average profit margin varies by only \textbf{\$31} across our three major markets (CA: \$3,026 vs. NY: \$2,995). In the context of automotive retail, a variance of 1\% is statistically negligible.

\textbf{Key Takeaways:}
\begin{itemize}
    \item \textbf{Universal Pricing Power:} We do not need to adjust our "floor prices" based on the state. The market in New York absorbs our pricing strategy just as readily as the market in Texas.
    \item \textbf{Simplified Expansion:} Since our margins do not degrade in "high cost" states (like CA/NY) or "high competition" states (like TX), we can confidently project these same margins for future expansion into new territories (e.g., Florida or Illinois).
    \item \textbf{Operational Efficiency:} We do not require regional managers to develop localized sales playbooks. A single, national sales training program is sufficient because the financial outcomes are identical everywhere.
\end{itemize}

\section{Efficiency: The "Days on Lot" Impact}
\begin{center}
\includegraphics{toyota_research-008}
\end{center}

Contrary to the expectation that "time kills deals," your data shows zero correlation between the time a car sits on the lot and the profit it generates. The trend line is virtually flat, indicating that the dealership network maintains strict pricing discipline and does not heavily discount aged inventory to move it.

\begin{itemize}
    \item \textbf{No "Panic Pricing":} The black trend line is horizontal. This proves that sales managers are not slashing prices on cars that have sat for 60+ days; they are holding out for the full margin regardless of age.
    \item \textbf{The Velocity Opportunity:} While protecting margins is good, this flat line reveals a hidden inefficiency. A \$3,000 profit earned on Day 90 is far less valuable than a \$3,000 profit earned on Day 10 due to capital tie-up. The goal should be to see a slight downward slope (tactical discounting) to increase turnover speed.
    \item \textbf{Consistent Execution:} The density of the red dots shows the bulk of sales occur consistently between 20-60 days with an evenly distributed profit spread, confirming that operational processes are highly standardized across the group.
\end{itemize}

We should implement an \textbf{"Age-Based" Pricing Policy}. Since profits currently do not drop over time, the dealership has room to be more aggressive. Introduce small, strategic price reductions at the 45-day mark. This will likely lower the per-unit profit slightly for aged cars (tilting the line down) but will drastically increase inventory turnover, freeing up cash to buy fresh, fast-selling stock.

\begin{Schunk}
\begin{Soutput}
        cor 
0.005805639 
\end{Soutput}
\begin{Soutput}
[1] 0.2456007
\end{Soutput}
\end{Schunk}

\textbf{ The "Holding Cost" Inefficiency}

Our Pearson correlation test returned a coefficient of \textbf{0.005}, confirming there is statistically \textbf{zero link} between how long a car sits on the lot and the profit it generates.

While this suggests excellent price discipline (managers are not "panic discounting" aged units), it reveals a critical operational flaw: \textbf{We are ignoring the Time Value of Money.}

\begin{itemize}
    \item \textbf{The Hidden Cost:} A \$2,500 profit earned on Day 90 is worth significantly less than a \$2,500 profit earned on Day 10, once floorplan interest and capital opportunity costs are factored in.
    \item \textbf{Stagnant Pricing:} The data proves we are "holding out" for full price even on stale inventory. This stagnates our turnover rate.
    \item \textbf{Strategic Pivot:} We should introduce an automated \textbf{"Velocity Discount"} (e.g., \$250 price drop every 15 days after Day 45). This will intentionally create a small negative correlation (lower profit on older cars) in exchange for a massive increase in volume and fresh inventory rotation.
\end{itemize}


\section{Customer \& Payment Insights}
\begin{center}
\includegraphics{toyota_research-010}
\end{center}

The dealership's profitability is remarkably resilient to financing methods. There is virtually no difference in gross profit between customers who pay cash, lease, or take out a loan.

\begin{itemize}
    \item \textbf{Uniform Profitability:} All three bars (Cash, Lease, Loan) are level at approximately \$3,000 per unit. This indicates that the dealership is not reliant on "back-end" finance reserve (profit from interest rate markups) to make deals work. The profit is baked into the vehicle price itself.
    \item \textbf{The "Cash Buyer" Myth:} A common industry belief is that cash buyers are less profitable because dealerships lose financing income. Your data disproves this for your specific group; cash buyers are generating the same net return as finance customers.
    \item \textbf{Operational Consistency:} This uniformity suggests that sales teams are strictly adhering to vehicle pricing floors, regardless of how the customer intends to pay.
\end{itemize}

We can \textbf{Simplify the Sales Process}. Since finance penetration does not drive higher gross profit per unit, the sales team can be "payment agnostic." Marketing efforts can freely target cash-rich buyers without fear of margin dilution. However, this also highlights a potential missed opportunity: the dealership may be under-optimizing its Finance \& Insurance (F\&I) products, as one would typically expect Lease/Loan deals to show slightly higher total margins due to finance reserves.

\section{Cross-Category Analysis}
\begin{Schunk}
\begin{Soutput}
	Pearson's Chi-squared test

data:  table_payment_car
X-squared = 9.3855, df = 6, p-value = 0.153
\end{Soutput}
\end{Schunk}
\textbf{ The "Universal Buyer" Behavior}

We performed a Chi-Squared Test of Independence to determine if specific vehicle segments attract different types of financial customers (e.g., do Truck buyers use loans more often than Sedan buyers?).

With a P-value of \textbf{0.153}, the data confirms that \textbf{Payment Method is Independent of Vehicle Choice}.

\begin{itemize}
    \item \textbf{No "Cash Segments":} We often assume that lower-priced vehicles (Sedans/Hatchbacks) attract more cash buyers while expensive vehicles (Trucks/SUVs) require financing. The data proves this is false. The ratio of Cash-to-Loan is consistent across our entire fleet.
    \item \textbf{Streamlined F\&I Strategy:} Since there is no "Lease-Heavy" or "Loan-Heavy" vehicle segment, our Finance Managers do not need specialized playbooks for different inventory types. A standardized financing presentation will work equally well for a Highlander customer as it does for a Camry customer.
    \item \textbf{Marketing Efficiency:} We do not need to waste ad spend targeting "Finance Special" ads specifically toward Truck buyers. The financial behavior of our customers is uniform, regardless of what they drive.
\end{itemize}

\subsection{Top 3 Dealership}
\includegraphics{toyota_research-012}



\begin{itemize}
    \item \textbf{The Visual:} The chart displays three distinct clusters of bars, one for each top dealership. Notably, the height of the bars within each cluster is nearly identical, indicating uniform profitability across payment methods.

    \item \textbf{The Data Breakdown:}
    \begin{itemize}
        \item \textbf{Toyota of Houston:} Cash deals (\$3,057) generated higher average profit than Loan (\$3,002) and Lease (\$2,993).
        \item \textbf{Brooklyn Toyota:} Cash (\$3,049) outperformed both Lease (\$2,984) and Loan (\$3,005).
        \item \textbf{Toyota of Los Angeles:} Lease (\$3,066) had a very slight edge, but Cash was immediately behind it (\$3,062).
    \end{itemize}

    \item \textbf{The Meaning:} This confirms that the "Payment Agnostic" trend observed earlier is not merely an aggregate average—it is a consistent operational reality at the specific dealership level for our largest volume leaders.

    \item \textbf{The Surprise:} Contrary to the industry standard where high-volume dealers rely heavily on "backend" finance profit (Loans) to boost margins, our data reveals that our biggest dealers are actually generating \textbf{more raw profit on Cash deals} in 2 out of 3 cases.
\end{itemize}

\textbf{The "Standardized Success" Audit}

We audited the performance of our top three volume leaders (Houston, Brooklyn, and Los Angeles) to detect if their profitability relied on specific financial products. The grouped bar chart reveals a striking operational consistency:

\begin{itemize}
    \item \textbf{The "Cash Premium" Reality:} Contrary to the industry norm where finance deals drive profit, our data shows that for \textbf{Toyota of Houston} and \textbf{Brooklyn Toyota}, Cash deals actually generate \textit{higher} average profit than Leases or Loans.
    \item \textbf{No "Finance Factory" Bias:} Large dealerships often pressure customers into financing to capture backend reserve. The flat profit distribution across these top dealers proves they are prioritizing the \textit{sale of the metal} over the \textit{sale of the loan}. This suggests a healthy, sustainable sales culture that doesn't rely on bank incentives to survive.
    \item \textbf{Scalability:} Because our top performers are achieving nearly identical margins (~\$3,000/unit) across all payment types, we have a "blueprint" for stability that can be rolled out to underperforming, smaller franchises.
\end{itemize}

\begin{center}
\includegraphics{toyota_research-013}

\end{center}
\subsection{ The "Border-Agnostic" Model}

Our regional profitability analysis returned a result of high strategic importance: \textbf{Geographic Irrelevance.}

As displayed in the chart, the average profit margin varies by only \textbf{\$31} across our three major markets (CA: \$3,026 vs. NY: \$2,995). In the context of automotive retail, a variance of 1\% is statistically negligible.

\textbf{Key Takeaways:}
\begin{itemize}
    \item \textbf{Universal Pricing Power:} We do not need to adjust our "floor prices" based on the state. The market in New York absorbs our pricing strategy just as readily as the market in Texas.
    \item \textbf{Simplified Expansion:} Since our margins do not degrade in "high cost" states (like CA/NY) or "high competition" states (like TX), we can confidently project these same margins for future expansion into new territories (e.g., Florida or Illinois).
    \item \textbf{Operational Efficiency:} We do not require regional managers to develop localized sales playbooks. A single, national sales training program is sufficient because the financial outcomes are identical everywhere.
\end{itemize}
\begin{center}
\includegraphics{toyota_research-014}
\end{center}

\textbf{ The "Payment Agnostic" Opportunity}

The inclusion of 95\% Confidence Interval error bars (the vertical lines on the chart) provides a definitive answer to the "Cash vs. Finance" debate. Because these error bars significantly overlap across all three categories, we have statistical proof that \textbf{profitability is identical} regardless of payment method.

This finding challenges two major operational norms:
\begin{enumerate}
    \item \textbf{Debunking the "Cash Bias":} Sales teams often de-prioritize cash buyers, assuming they yield lower margins due to a lack of finance reserve. The data proves this is a fallacy. A cash buyer is just as valuable as a finance buyer.
    \item \textbf{The F\&I Gap:} While equal profitability is "good" news for cash deals, it is \textbf{concerning news for finance deals}. Typically, Loan/Lease deals should yield \textit{higher} total profit due to backend products (warranties, gap insurance). The fact that they are merely equal to cash deals suggests our Finance \& Insurance (F\&I) department may be under-penetrating on backend product sales.
\end{enumerate}

\textbf{Strategic Recommendation:} Remove any sales incentives that favor finance customers over cash customers. Simultaneously, audit F\&I performance to understand why finance deals are not outperforming cash deals as per industry standards.

\section{Strategic Pricing Model (Regression)}
\begin{center}
\begin{Schunk}
\begin{Soutput}
Call:
lm(formula = SalePrice ~ Year + Mileage + CarType + DaysOnLot, 
    data = df)

Residuals:
     Min       1Q   Median       3Q      Max 
-16665.2  -2689.6     -5.2   2711.3  17282.1 

Coefficients:
               Estimate Std. Error  t value Pr(>|t|)    
(Intercept)  -1.783e+06  1.174e+04 -151.898   <2e-16 ***
Year          8.979e+02  5.804e+00  154.713   <2e-16 ***
Mileage      -6.682e-02  1.013e-03  -65.948   <2e-16 ***
CarTypeSedan -1.548e+01  6.463e+01   -0.239    0.811    
CarTypeSUV   -2.372e+01  6.448e+01   -0.368    0.713    
CarTypeTruck  3.686e+01  6.462e+01    0.570    0.568    
DaysOnLot    -2.762e-01  1.338e+00   -0.206    0.836    
---
Signif. codes:  0 '***' 0.001 '**' 0.01 '*' 0.05 '.' 0.1 ' ' 1

Residual standard error: 3999 on 39993 degrees of freedom
Multiple R-squared:  0.6146,	Adjusted R-squared:  0.6145 
F-statistic: 1.063e+04 on 6 and 39993 DF,  p-value: < 2.2e-16
\end{Soutput}
\end{Schunk}
\end{center}

The pricing model is incredibly simple. Only Year and Mileage matter. Contrary to expectations, broad categories like "Car Type" (SUV vs. Sedan) and operational metrics like "Days on Lot" have zero statistical impact on the final sale price in this specific dataset.\\\


\textbf{1. The "Golden Rules" of Valuation (Statistically Significant)}
The model identified two variables with extreme certainty ($p < 2e-16$) that drive vehicle price:

\begin{itemize}
    \item \textbf{The Age Premium:} For every year newer a vehicle is, its market value increases by approximately \$898.
    \item \textbf{The Mileage Penalty:} For every single mile driven, the vehicle loses roughly \$0.07 (7 cents) in value.
\end{itemize}

\textbf{2. The Surprising "Non-Factors" (Insignificant)}

\begin{itemize}
    \item \textbf{Car Type is Irrelevant:} The P-values for Sedan (0.81), SUV (0.71), and Truck (0.57) are huge. This means that, statistically, a 2018 SUV with 50k miles sells for roughly the same price as a 2018 Truck with 50k miles in this dataset. The "segment premium" we assume exists does not appear in the math.
    \item \textbf{Time Doesn't Decay Price:} The \texttt{DaysOnLot} variable is statistically insignificant. This confirms our earlier "Scatter Plot" finding: a car sold on Day 90 commands the same price as one sold on Day 1.
\end{itemize}

\textbf{3. Model Reliability}\\\

\textbf{R-Squared (0.61):} The model explains 61.5\% of the price variation. This is a solid baseline, but it means \textasciitilde 40\% of the price is determined by factors not in this data (likely specific trim levels like "Limited vs. LE", vehicle condition, or color).

\textbf{Strategic Recommendation:} Simplify the Pricing Grid. The dealership can automate 60\% of its initial pricing strategy using a simple formula: 
\[ Base Price + (\$900 \times Year) - (\$0.07 \times Mileage) \]
Stop adding manual "gut check" premiums for SUVs or discounts for aged inventory, as the data proves the market does not statistically penalize or reward these factors in your current sales history.

\begin{Schunk}
\begin{Soutput}
              GVIF Df GVIF^(1/(2*Df))
Year      1.571167  1        1.253462
Mileage   1.571311  1        1.253519
CarType   1.000344  3        1.000057
DaysOnLot 1.000101  1        1.000051
\end{Soutput}
\end{Schunk}

\textbf{Executive Insight: Model Integrity Validation}

To ensure our pricing formula is reliable, we performed a Variance Inflation Factor (VIF) analysis. This tests for "Multi-Collinearity"—a statistical error where variables confuse the model (e.g., if "Year" and "Mileage" were too closely linked, the model wouldn't know which one was actually driving the price).

\textbf{The Results are Definitive:}
\begin{itemize}
    \item \textbf{High Stability:} All VIF scores are below 1.6 (well below the warning threshold of 5.0). This confirms that our variables are statistically distinct.
    \item \textbf{Trustworthy Premiums:} Because the model is stable, the specific dollar values identified earlier (the \$898 premium per Year and the \$0.07 penalty per Mile) are accurate reflections of market behavior, not mathematical artifacts.
    \item \textbf{Actionable Data:} We can safely program these specific values into our inventory management software without fear of distortion.
\end{itemize}
\end{document}
